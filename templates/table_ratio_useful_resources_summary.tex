\renewcommand{\arraystretch}{1.2}
\begin{table*}
	\begin{center}
		\resizebox{\linewidth}{!}{\begin{tabular}{|c|c|c|c|c|c|c|c|c|c|c|}
			\hline
            Approach/ Query Template & D1 & D2 & D3 & D4 & D5 & D6 & D7 & S1 & S4 & S5 \\
			\hline
            shape index (\%) & {} & {} & {} & {} & {} & {} & {} & {} & {} & {} \\
            \hline
			type index (\%) & {} & {} & {} & {} & {} & {} & {} & {} & {} & {} \\
			\hline
		\end{tabular}}
	\end{center}
	\caption{
		\textbf{The percentage of query-relevant resources.}
        For most queries, the percentage of query-relevant resources is low (cell values are shown as $\text{avg}^{\text{max}}_{\text{min}}$). 
		Shape index generally matches or outperforms the type index, except for templates D6, D7, and S4.
		The difference for S4 is more pronounced, reaching 100\% with the Type Index compared to only 6\% with the shape index.
		}
	\label{tab:ratioUsefulResources}
\end{table*}